\documentclass[UTF8]{ctexbook}
\usepackage{graphicx}
\usepackage{booktabs}
\usepackage{color}
\usepackage{multirow}
\usepackage{fontspec}
\usepackage{amsmath}
\usepackage{amssymb}
\usepackage{tikz}
\usepackage{subfigure}
\usepackage{amsthm,amsfonts,mathtools}
%\usepackage[noend]{algpseudocode}
%\usepackage{algorithmicx,algorithm}
\usepackage[ruled]{algorithm2e}
\usepackage{textpos}
\usepackage{url}
\usepackage{algorithmic}

\newtheorem{property}{性质}
% \setmainfont{LMRomanUnsl10-Regular}
% \setmainfont{Noto Sans Mono CJK JP}
% \setmainfont{黑体}

\begin{document}
\title{优化算法的学习}

\author{李鑫}

\date{\today}

\maketitle
\tableofcontents

\mainmatter %% 表示文章的正文部分,在生成目录后将从第一页开始

\part{理论部分}

\chapter{最优性的表示}

\section{凸集的分离定理}

\section{Fakars引理}



\section{KKT条件}





\end{document}

%%% Local Variables:
%%% mode: latex
%%% TeX-master: t
%%% TeX-engine: xetex
%%% End:
