\documentclass[UTF8]{ctexbook}
\usepackage{graphicx}
\usepackage{booktabs}
\usepackage{color}
\usepackage{multirow}
\usepackage{fontspec}
\usepackage{amsmath}
\usepackage{amssymb}
\usepackage{tikz}
\usepackage{subfigure}
\usepackage{amsthm,amsfonts,mathtools}
%\usepackage[noend]{algpseudocode}
%\usepackage{algorithmicx,algorithm}
\usepackage[ruled]{algorithm2e}
\usepackage{textpos}
\usepackage{url}
\usepackage{algorithmic}

\newtheorem{property}{性质}

% \setmainfont{LMRomanUnsl10-Regular}
% \setmainfont{Noto Sans Mono CJK JP}
% \setmainfont{黑体}

\setCJKmainfont{Microsoft YaHei}
\setmainfont{Noto Mono}


\begin{document}
\title{优化算法的学习}

\author{李鑫}

\date{\today}

\maketitle
\tableofcontents

\mainmatter %% 表示文章的正文部分,在生成目录后将从第一页开始

\part{理论部分}

\chapter{最优性的表示}

\section{如何表示最优性}

我的理解:从局部最优点到任意可行区域内的点,目标函数变得更差(最小化目标函数的就
是变大)。那么如何刻画 \textsf{可行区域} 和 \textsf{目标函数变得更差}。

首先是可行区域,就是控制变量在什么样子的集合里面。各种问题应该有各种奇奇怪怪的集
合了,所以我们想通过一些数学定义,定义出一些性质比较好的集合(一次,二次)。

最简单的集合应该是空集和一个元素的集合,但是在实际生活中都没什么意义。进一步就是
线或者说是具有线性的集合,在一维里面就是过原点的线,二维里面是过原点的平面。但是
这样的集合太大,比如整个二维平面,而我们只想里面的一部分,并且保持一定的线性。从
这个想法出发,简单的把平面分成几份,每一份都是类似三角形无限伸长。我们把这样的东
西称为锥,用数学的语言来描述就是
\begin{align}
 \alpha  ,    \beta \in C, \ a,  b \ge 0 \Rightarrow \  a \alpha + b \beta \in C
\end{align}
进一步的,我们很多东西都是有限的,不可能真的有无限伸长的东西,所以就要把这个集合
封闭起来。如何封闭,这样封闭想保持什么样的性质,是需要我们事前想好的。

那么脑袋里首先映入的就是把两个锥对着放一起,组成一个集合,菱形一样的玩意。但是是
不是太特殊了,无法拓展使用。想要封闭,肯定要想到线段了,对应这线性。线段这个东西
用数学的语言来描述就是
\begin{align}
1 \ge  \lambda \ge 0, a, b \in C  \Rightarrow b + \lambda ( a-b ) \in C  \Rightarrow \  \lambda a + ( 1-\lambda ) b \in C 
\end{align}
这样就会形成最为重要的概念,凸集。

下面再来考虑使得目标函数变得更差这样的刻画方法。说白了,目标函数是函数,所以这里
我们得讨论下目标函数的函数特性。因为是局部最优,即在一个小范围内的最优。看到局部,
小范围什么的第一反应当然是泰勒展开,用线性项近似我们原来的函数(无论原来函数是什
么样子),这样就一下子大大简化了范围。
\begin{align}
  f(x) = f( x_0 ) + \nabla f( x_0 )  <x, x_0  >
\end{align}
如果$x_0$是最小值,那么在局部,就
\begin{align}
  f(x) - f( x_0 ) = \nabla f( x_0 ) < x, x_0 > \  \ge 0
\end{align}
即任意可行区域内的点与最小点的方向,和最小点处的梯度,夹角小于$90^{ \circ }$。

\section{凸集的基本性质}

\subsection{数学分析的几个基本定理}
映射:A的任何元素,都有唯一的B 中元素与之对应(\url{https://www.bilibili.com/video/BV1YA411s7tY})。

\subsubsection{单调有界定理}

\subsubsection{闭区间套定理}

\subsubsection{Bolzano-Weierstrass Theorem}
有界数列必有收敛子列

\subsubsection{Cauchy收敛准则}


\subsubsection{确界原理}

 
\subsubsection{有限覆盖定理}


\subsection{投影定理}
我的理解:由于线性,就会天然存在三角不等式来保证唯一性。


\subsection{点与凸集的分离}

我的理解:一个点,一个凸集合一定可以找到一条线把这两个东西分开。



\subsection{支撑超平面定理}

我的理解:一根棒棒翘起来了凸集

\section{Farkas引理}
我的理解:一个点,要么在凸锥外面,要么在凸锥里面。




设矩阵$A_{ m \times n} $和$n$维向量$c$,下面问题有且只有一个有解
\begin{itemize}
\item $A x \leq 0, \ c^T x > 0 $.
\item $A^T y = c, y \ge 0$.
\end{itemize}

\section{凸函数}
凸函数的定义域一定是凸集合,乍一看凭什么啊,怎么得到的啊,晕糊糊的。其实应该从凸
函数的起源说起,为什么我们要搞这样的一个函数出来,因为我们想简单化问题。如何简单
化呢,就是我得到一个局部极小点就是全局最小点。如何保证这样的性质呢,说起来也简单,
联想我们初高中学过的如果一个函数的导数在图形上只有一个零点,那么最值就那一个。如
何保证导数就一个零点呢,假设导数是单调的。如果是多维的,即梯度是单调函数
\begin{align}
  < \nabla f(x) - \nabla f(y), x - y > \ \ge 0
\end{align}

通过这个性质,我们可以得到函数与它一阶近似的关系
\begin{align}
  \begin{split}
    \int_0^1 \frac{d}{dt} f( x + t ( y - x ) ) dt & = f( x + t ( y - x ) ) |_0^1 =
    f(y) - f(x) \\
    & = \int_0^1 { \nabla f( x + t ( y-x ) )  }^T ( y-x )    
  \end{split}
\end{align}
从而得到
\begin{align}
  f(y) = f(x) + \int_0^1 { \nabla f( x + t ( y-x ) )  }^T ( y-x )
\end{align}
这里设$g(t) = { \nabla f( x + t ( y-x ) )  }^T ( y-x )$ 可以得到
\begin{align}
  \begin{split}
    g(t) - g(0) & = { \nabla f( x + t ( y-x ) )  }^T ( y-x ) - { \nabla f( x )
    }^T ( y-x ) \\
    & = <{ \nabla f( x + t ( y-x ) )  }^T - \nabla { f( x )  }^T , y -x > \\
    & = \frac{1}{t} <{ \nabla f( x + t ( y-x ) )  }^T - \nabla { f( x )  }^T ,
    t(y -x) > \\
    & \geq 0
  \end{split}
\end{align}
这样$g(0)$是$g(t)$在$t \in [0, 1]$的最小值,可得
\begin{align}
  \begin{split}
    f(y) & = f(x) + \int_0^1 { \nabla f( x + t ( y-x ) )  }^T ( y-x ) \\
    & \geq f(x) + \int_0^1 { \nabla f(x) }^T ( y-x ) \\
    & = f(x) + < \nabla f(x), y-x >
  \end{split}
\end{align}

这就是我们常见的凸函数与其一阶近似的关系,再由这个关系,我们可以推导出书上直接给
的定义,设$a = x + t ( y-x ) $
\begin{align}
  \begin{split}
    & f(x) \geq f(a) + { \nabla f(a)  }^T ( x - a ) \\
    & f(y) \geq f(a) + { \nabla f(a) }^T ( y - a)  \\
    & ( 1-t ) f(x) \geq ( 1-t ) f(a) + ( 1-t ) { \nabla f(a)  }^T ( x - a ) \\
    & t f(y) \geq t f(a) + t { \nabla f(a) }^T ( y - a)  
  \end{split}
\end{align}
相加可得到我们“熟悉”的定义
\begin{align}
  ( 1-t ) f(x) + t f(y) \geq  f( x + t ( y-x ) ) = f( t y + ( 1-t ) x )
\end{align}

\section{无约束问题}

\subsection{一阶算法复杂度}

\subsection{二阶算法复杂度}


\section{KKT条件}
局部最优解的特征(\url{https://www.bilibili.com/video/BV1Tt4y127U1})。




\end{document}

%%% Local Variables:
%%% mode: latex
%%% TeX-master: t
%%% TeX-engine: xetex
%%% End:
