\documentclass[UTF8, aspectratio=169, 9pt]{ctexbeamer}

\usepackage{graphicx}
\usepackage{booktabs}
\usepackage{color}
\usepackage{multirow}
\usepackage{fontspec}
\usepackage{amsmath}
\usepackage{amssymb}
\usepackage{tikz}
\usepackage{subfigure}
\usepackage{amsthm,amsfonts,mathtools}
%\usepackage[noend]{algpseudocode}
%\usepackage{algorithmicx,algorithm}
\usepackage[ruled]{algorithm2e}
\usepackage{textpos}
\usepackage{url}
\usepackage{algorithmic}

\newtheorem{property}{性质}
% \setmainfont{LMRomanUnsl10-Regular}
% \setmainfont{Noto Sans Mono CJK JP}
% \setmainfont{黑体}

% \useoutertheme{miniframes}
\usetheme{Goettingen}
% \usetheme{Madrid}
% \usetheme{Berkeley}
\title{自用复习}
\subtitle{近似点算法学习}
\author{李鑫}
\institute{吉林大学}
\date{\today}
\begin{document}
\frame{\titlepage}
\frame{\tableofcontents}

\section{线性代数基础}

\subsection{对称矩阵}
\begin{frame}
  \frametitle{对称矩阵定义及其性质}
  \begin{definition}[对称矩阵]
  矩阵$A \in \mathbb{R}^{n \times n}$ 是对称的: $ A^T = A $
  \end{definition}
  两个基本性质
  
  \begin{property}[实对称矩阵]
    \begin{enumerate}
    \item 实对称矩阵所有特征值为实数
    \item 不同特征值的特征向量正交
    \end{enumerate}
  \end{property}

  复习下复数的东西
  \begin{itemize}
  \item 先来看看什么叫做复数矩阵$A \in \mathbb{C}^{n \times n}$. 对于复数定义为$z = x
  + y i$其中$ i = \sqrt {-1} $. 那么对应的$\mathbb{C}^n$就是每一个元素都是复数的
  向量,而$\mathbb{C}^{ n \times n }$就是每一个元素都是复数的矩阵.
  \item 我们记复数的共轭是$z^{*} = x - y i$. 对于复数向量$u \in \mathbb{C}^n$和
    复数矩阵$A \in \mathbb{C}^{n \times n}$, 他们的共轭是他们每个元素的共轭.
  \item $u \in \mathbb{C}^n $的共轭是$(u_i)^{*} = ( u^{*} )_i$.
  \item $A \in \mathbb{C}^{n \times n}$的共轭是$(A_{ij})^{*} = (A^{*})_{ij}$.
  \item 性质
    \begin{itemize}
    \item $u=v \Leftrightarrow u^{*} = v^{*}$.
    \item $A,B \in \mathbb{C}^{n \times n}, \,  A=B  \Leftrightarrow  A^{*}
      = B^{*} $.
    \item $(Au)^{*} = A^{*} u^{*}$ 和$(A^{*})^{T} = (A^{T})^{*}$.
    
    \end{itemize}
  \end{itemize}
  
  
  
\end{frame}

\begin{frame}
  \frametitle{对称矩阵定义及其性质}
  \begin{proof}[实对称矩阵基本性质的证明]
    \begin{enumerate}
    \item 对于特征值$\lambda$和特征向量$u$,有
      \begin{align}
        \label{eq:01}
        Au = \lambda u
      \end{align}
      对两边同时取共轭可得到
      \begin{align}
        \label{eq:02}
        A^{*} u^{*} = \lambda^{*} u^{*} = A u^{*}
      \end{align}
      对公式 (\ref{eq:01}) 两边同时乘以$( u^{*} )^{T}$得到
      \begin{align}
        \begin{split}
          \lambda (u^{*})^T u &=  ( u^{*} )^T (Au) = ( (u^{*})^T A ) u \\
          &= ( A^T u^{*} )^T u \\
          &= ( A u^{*} )^T u \\
          &= ( \lambda^{*} u^{*} )^T u = \lambda^{*} ( u^{*} )^T u
        \end{split}
      \end{align}
      得到$(\lambda - \lambda^{*} ) (u^{*})^T u = 0$,证明完毕.
    \end{enumerate}
  \end{proof}  
\end{frame}

\begin{frame}
  \frametitle{对称矩阵定义及其性质}
  \begin{proof}[实对称矩阵基本性质的证明]
    \begin{enumerate}[2]
    \item 对于不同特征值$\lambda \neq u$对于的特征向量$x \neq y$,有
      \begin{align}
        \begin{split}
          <A x, y> &= \lambda <x, y>  \\
          <A x, y> &= <x, A y> = u <x, y> \\
          ( \lambda - u ) <x, y> &= 0
        \end{split}
      \end{align}
      由于$ \lambda \neq u $得到$<x, y> = 0$.
    \end{enumerate}
  \end{proof}
 \end{frame}


\section{背景}
\subsection{知乎笔记}
\begin{frame}
  \frametitle{基础知识}
  集合$\mathcal{C}$是闭集,如果它包含边界,即
  $$
  x^k \in \mathcal{C}, \, x^k \rightarrow \bar{x} \quad \Rightarrow \quad \bar{x}
  \in \mathcal{C}
  $$
  保持闭集的操作
  \begin{itemize}
  \item 闭集的交集还是闭集.
  \item 有限个闭集的并集还是闭集.
  \item 如果$\mathcal{C}$是闭集,则线性映射的原象也是闭集,也就是$\{ x|Ax \in
    \mathcal{C} \}$是闭集合
  \end{itemize}
\end{frame}

\thispagestyle{empty} 
\begin{frame}{}
  \centering \Huge
  \emph{Thank You}
\end{frame}
\end{document}
%%% Local Variables:
%%% mode: latex
%%% TeX-master: t
%%% TeX-engine: xetex
%%% End:
